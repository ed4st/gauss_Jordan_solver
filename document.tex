\documentclass{article}\usepackage[]{graphicx}\usepackage[]{color}
% maxwidth is the original width if it is less than linewidth
% otherwise use linewidth (to make sure the graphics do not exceed the margin)
\makeatletter
\def\maxwidth{ %
  \ifdim\Gin@nat@width>\linewidth
    \linewidth
  \else
    \Gin@nat@width
  \fi
}
\makeatother

\definecolor{fgcolor}{rgb}{0.345, 0.345, 0.345}
\newcommand{\hlnum}[1]{\textcolor[rgb]{0.686,0.059,0.569}{#1}}%
\newcommand{\hlstr}[1]{\textcolor[rgb]{0.192,0.494,0.8}{#1}}%
\newcommand{\hlcom}[1]{\textcolor[rgb]{0.678,0.584,0.686}{\textit{#1}}}%
\newcommand{\hlopt}[1]{\textcolor[rgb]{0,0,0}{#1}}%
\newcommand{\hlstd}[1]{\textcolor[rgb]{0.345,0.345,0.345}{#1}}%
\newcommand{\hlkwa}[1]{\textcolor[rgb]{0.161,0.373,0.58}{\textbf{#1}}}%
\newcommand{\hlkwb}[1]{\textcolor[rgb]{0.69,0.353,0.396}{#1}}%
\newcommand{\hlkwc}[1]{\textcolor[rgb]{0.333,0.667,0.333}{#1}}%
\newcommand{\hlkwd}[1]{\textcolor[rgb]{0.737,0.353,0.396}{\textbf{#1}}}%
\let\hlipl\hlkwb

\usepackage{framed}
\makeatletter
\newenvironment{kframe}{%
 \def\at@end@of@kframe{}%
 \ifinner\ifhmode%
  \def\at@end@of@kframe{\end{minipage}}%
  \begin{minipage}{\columnwidth}%
 \fi\fi%
 \def\FrameCommand##1{\hskip\@totalleftmargin \hskip-\fboxsep
 \colorbox{shadecolor}{##1}\hskip-\fboxsep
     % There is no \\@totalrightmargin, so:
     \hskip-\linewidth \hskip-\@totalleftmargin \hskip\columnwidth}%
 \MakeFramed {\advance\hsize-\width
   \@totalleftmargin\z@ \linewidth\hsize
   \@setminipage}}%
 {\par\unskip\endMakeFramed%
 \at@end@of@kframe}
\makeatother

\definecolor{shadecolor}{rgb}{.97, .97, .97}
\definecolor{messagecolor}{rgb}{0, 0, 0}
\definecolor{warningcolor}{rgb}{1, 0, 1}
\definecolor{errorcolor}{rgb}{1, 0, 0}
\newenvironment{knitrout}{}{} % an empty environment to be redefined in TeX

\usepackage{alltt}
\usepackage[utf8]{inputenc}
\usepackage{amsmath}
\usepackage{mathtools}
\usepackage{enumerate}% http://ctan.org/pkg/enumerate
\usepackage[thinc]{esdiff}	
\usepackage{amsfonts}
\DeclareUnicodeCharacter{0301}{\'{i}}
\title{Documento de uso}
\author{Edgar Steven Baquero Acevedo}
\date{Septiembre 9, 2020}
\IfFileExists{upquote.sty}{\usepackage{upquote}}{}
\begin{document}

\maketitle

\section*{Primera parte}
Para la primera parte tenemos que llamar la función \texttt{main\_function} ubicada en el archivo \texttt{elementary\_operations.R}. Así, tomando como ejemplo la siguiente martriz:
\begin{equation*}
m=
\begin{pmatrix*}[l]
1 & 2 & 3 & -1\\
4 & 5 & 6 & 3\\
7 & 8 & 9 & 5
\end{pmatrix*}
\end{equation*}
para ingresarla nos pedirá que ingresemos el tamaño de la matriz separado por coma. En nuestro caso, la matriz es $3\times 4$:
\begin{knitrout}
\definecolor{shadecolor}{rgb}{0.969, 0.969, 0.969}\color{fgcolor}\begin{kframe}
\begin{alltt}
\hlcom{# > main_function()}
\hlcom{# Enter the size of the mxn matrix separated by coma (,):3,4}
\end{alltt}
\end{kframe}
\end{knitrout}
Luego, nos pedirá los datos de las filas separados por coma:
\begin{knitrout}
\definecolor{shadecolor}{rgb}{0.969, 0.969, 0.969}\color{fgcolor}\begin{kframe}
\begin{alltt}
\hlcom{# [1] "Enter the data by rows, separated by coma (,)"}
\hlcom{# Row 1: 1,2,3,-1}
\hlcom{# Row 2: 4,5,6,3}
\hlcom{# Row 3: 7,8,9,5}
\end{alltt}
\end{kframe}
\end{knitrout}
Una vez hecho esto, nos aparecerá un menú:
\begin{knitrout}
\definecolor{shadecolor}{rgb}{0.969, 0.969, 0.969}\color{fgcolor}\begin{kframe}
\begin{alltt}
\hlcom{# [1] "1. Intercambiar filas."}
\hlcom{# [1] "2. Multiplicar una fila por un escalar distinto de 0."}
\hlcom{# [1] "3. Remplazar una fila por un múltiplo escalar de otra fila."}
\hlcom{# [1] "4. Salir"}
\end{alltt}
\end{kframe}
\end{knitrout}
En nuestro caso, seleccionaremos 1, e intercambiamos la fila 2 y 3:
\begin{knitrout}
\definecolor{shadecolor}{rgb}{0.969, 0.969, 0.969}\color{fgcolor}\begin{kframe}
\begin{alltt}
\hlcom{# ¿Qué operación quieres realizar?:1}
\hlcom{# ingresa las filas a intercambiar, separado por coma:2,3}
\hlcom{# [1] "¡operación realizada!"}
\hlcom{#      [,1] [,2] [,3] [,4]}
\hlcom{# [1,]    1    2    3   -1}
\hlcom{# [2,]    7    8    9    5}
\hlcom{# [3,]    4    5    6    3}
\end{alltt}
\end{kframe}
\end{knitrout}
Efectivamente, nuestra matriz tiene la forma:
\begin{equation*}
\begin{pmatrix*}[l]
1 & 2 & 3 & -1\\
7 & 8 & 9 & 5\\
4 & 5 & 6 & 3
\end{pmatrix*}
\end{equation*}
Por motivos prácticos volveremos a intercambiar la mismas filas (2 y 3): 
\begin{knitrout}
\definecolor{shadecolor}{rgb}{0.969, 0.969, 0.969}\color{fgcolor}\begin{kframe}
\begin{alltt}
\hlcom{# [1] "1. Intercambiar filas."}
\hlcom{# [1] "2. Multiplicar una fila por un escalar distinto de 0."}
\hlcom{# [1] "3. Remplazar una fila por un múltiplo escalar de otra fila."}
\hlcom{# [1] "4. Salir"}
\hlcom{# ¿Qué operación quieres realizar?:1}
\hlcom{# ingresa las filas a intercambiar, separado por coma:2,3}
\hlcom{# [1] "¡operación realizada!"}
\hlcom{#      [,1] [,2] [,3] [,4]}
\hlcom{# [1,]    1    2    3   -1}
\hlcom{# [2,]    4    5    6    3}
\hlcom{# [3,]    7    8    9    5}
\hlcom{# [1] "1. Intercambiar filas."}
\hlcom{# [1] "2. Multiplicar una fila por un escalar distinto de 0."}
\hlcom{# [1] "3. Remplazar una fila por un múltiplo escalar de otra fila."}
\hlcom{# [1] "4. Salir"}
\end{alltt}
\end{kframe}
\end{knitrout}
De nuevo, obtenemos nuestra matris original $m$. 

Ahora con la opción 2, multiplicaremos la fila 2 por 2. Esto es:
\begin{knitrout}
\definecolor{shadecolor}{rgb}{0.969, 0.969, 0.969}\color{fgcolor}\begin{kframe}
\begin{alltt}
\hlcom{# [1] "1. Intercambiar filas."}
\hlcom{# [1] "2. Multiplicar una fila por un escalar distinto de 0."}
\hlcom{# [1] "3. Remplazar una fila por un múltiplo escalar de otra fila."}
\hlcom{# [1] "4. Salir"}
\hlcom{# ¿Qué operación quieres realizar?:2}
\hlcom{# ingresa la fila  y el escalar, separado por coma:2,2}
\hlcom{# [1] "¡operación realizada!"}
\hlcom{#      [,1] [,2] [,3] [,4]}
\hlcom{# [1,]    1    2    3   -1}
\hlcom{# [2,]    8   10   12    6}
\hlcom{# [3,]    7    8    9    5}
\end{alltt}
\end{kframe}
\end{knitrout}
Efectivamente, tenemos la matriz:
\begin{equation*}
\begin{pmatrix*}[l]
1 & 2 & 3 & -1\\
8 & 10 & 12 & 6\\
7 & 8 & 9 & 5
\end{pmatrix*}
\end{equation*}
 Ahora, con la operación 3, empezaremos a reducirla a forma escalonada:
 Hagamos $F_2 = F_2 -8F_1$:
\begin{knitrout}
\definecolor{shadecolor}{rgb}{0.969, 0.969, 0.969}\color{fgcolor}\begin{kframe}
\begin{alltt}
\hlcom{# [1] "1. Intercambiar filas."}
\hlcom{# [1] "2. Multiplicar una fila por un escalar distinto de 0."}
\hlcom{# [1] "3. Remplazar una fila por un múltiplo escalar de otra fila."}
\hlcom{# [1] "4. Salir"}
\hlcom{# ¿Qué operación quieres realizar?:3}
\hlcom{# ingresa la fila a la que sumarás,}
\hlcom{#                    la fila que sumarás y el múltiplo escalar de esta última,}
\hlcom{#                    separado por coma:2,1,-8}
\hlcom{# [1] "¡operación realizada!"}
\hlcom{# [1] -8}
\hlcom{#      [,1] [,2] [,3] [,4]}
\hlcom{# [1,]    1    2    3   -1}
\hlcom{# [2,]    0   -6  -12   14}
\hlcom{# [3,]    7    8    9    5}
\end{alltt}
\end{kframe}
\end{knitrout}
Ahora haremos $F_3 = F_3 - 7F_1$:
\begin{knitrout}
\definecolor{shadecolor}{rgb}{0.969, 0.969, 0.969}\color{fgcolor}\begin{kframe}
\begin{alltt}
\hlcom{# [1] "1. Intercambiar filas."}
\hlcom{# [1] "2. Multiplicar una fila por un escalar distinto de 0."}
\hlcom{# [1] "3. Remplazar una fila por un múltiplo escalar de otra fila."}
\hlcom{# [1] "4. Salir"}
\hlcom{# ¿Qué operación quieres realizar?:3}
\hlcom{# ingresa la fila a la que sumarás,}
\hlcom{#                    la fila que sumarás y el múltiplo escalar de esta última,}
\hlcom{#                    separado por coma:3,1,-7}
\hlcom{# [1] "¡operación realizada!"}
\hlcom{# [1] -7}
\hlcom{#      [,1] [,2] [,3] [,4]}
\hlcom{# [1,]    1    2    3   -1}
\hlcom{# [2,]    0   -6  -12   14}
\hlcom{# [3,]    0   -6  -12   12}
\end{alltt}
\end{kframe}
\end{knitrout}
Por último haremos $F_3 = F_3 - F_2$ para obtener la forma escalonada:
\begin{knitrout}
\definecolor{shadecolor}{rgb}{0.969, 0.969, 0.969}\color{fgcolor}\begin{kframe}
\begin{alltt}
\hlcom{# [1] "1. Intercambiar filas."}
\hlcom{# [1] "2. Multiplicar una fila por un escalar distinto de 0."}
\hlcom{# [1] "3. Remplazar una fila por un múltiplo escalar de otra fila."}
\hlcom{# [1] "4. Salir"}
\hlcom{# ¿Qué operación quieres realizar?:3}
\hlcom{# ingresa la fila a la que sumarás,}
\hlcom{#                    la fila que sumarás y el múltiplo escalar de esta última,}
\hlcom{#                    separado por coma:3,2,-1}
\hlcom{# [1] "¡operación realizada!"}
\hlcom{# [1] -1}
\hlcom{#      [,1] [,2] [,3] [,4]}
\hlcom{# [1,]    1    2    3   -1}
\hlcom{# [2,]    0   -6  -12   14}
\hlcom{# [3,]    0    0    0   -2}
\hlcom{# [1] "La matriz se redujo a forma escalonada:"}
\hlcom{#      [,1] [,2] [,3] [,4]}
\hlcom{# [1,]    1    2    3   -1}
\hlcom{# [2,]    0   -6  -12   14}
\hlcom{# [3,]    0    0    0   -2}
\end{alltt}
\end{kframe}
\end{knitrout}
Donde confirmamos que el programa verifica que se llegó a la forma escalonada deseada:
 \begin{equation*}
\begin{pmatrix*}[l]
1 & 2 & 3 & -1\\
0 & -6 & -12 & 14\\
0 & 0 & 0 & -2
\end{pmatrix*}
\end{equation*}



\section*{Segunda parte}
Para la segunda parte, es necesario estar en el archivo \texttt{reduce\_to\_echelon.R}, y ejecutar la función \texttt{main\_function2} presente allí mismo:
tomando el mismo ejemplo:
\begin{equation*}
m=
\begin{pmatrix*}[l]
1 & 2 & 3 & -1\\
4 & 5 & 6 & 3\\
7 & 8 & 9 & 5
\end{pmatrix*}
\end{equation*}
para ingresarla nos pedirá que ingresemos el tamaño de la matriz separado por coma. En nuestro caso, la matriz es $3\times 4$:
\begin{knitrout}
\definecolor{shadecolor}{rgb}{0.969, 0.969, 0.969}\color{fgcolor}\begin{kframe}
\begin{alltt}
\hlcom{# main_function2()}
\hlcom{# Enter the size of the mxn matrix separated by coma (,):3,4}
\end{alltt}
\end{kframe}
\end{knitrout}
De nuevo, llenamos la matriz por filas:
\begin{knitrout}
\definecolor{shadecolor}{rgb}{0.969, 0.969, 0.969}\color{fgcolor}\begin{kframe}
\begin{alltt}
\hlcom{# Enter the size of the mxn matrix separated by coma (,):3,4}
\hlcom{# [1] "Enter the data by rows, separated by coma (,)"}
\hlcom{# Row 1: 1,2,3,-1}
\hlcom{# Row 2: 4,5,6,3}
\hlcom{# Row 3: 7,8,9,5}
\hlcom{#      [,1] [,2] [,3] [,4]}
\hlcom{# [1,]    1    2    3   -1}
\hlcom{# [2,]    0   -3   -6    7}
\hlcom{# [3,]    0    0    0   -2}
\end{alltt}
\end{kframe}
\end{knitrout}
Y efectivamente, una vez llenada, nos redujo a la matriz escalonada reducida:
\begin{equation*}
\begin{pmatrix*}[l]
1 & 2 & 3 & -1\\
0 & -3 & -6 & 7\\
0 & 0 & 0 & -2
\end{pmatrix*}
\end{equation*}


\end{document}
